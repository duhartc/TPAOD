\documentclass[a4paper, 10pt, french]{article}
% Préambule; packages qui peuvent être utiles
   \RequirePackage[T1]{fontenc}        % Ce package pourrit les pdf...
   \RequirePackage{babel,indentfirst}  % Pour les césures correctes,
                                       % et pour indenter au début de chaque paragraphe
   \RequirePackage[utf8]{inputenc}   % Pour pouvoir utiliser directement les accents
                                     % et autres caractères français
   \RequirePackage{lmodern,tgpagella} % Police de caractères
   \textwidth 17cm \textheight 25cm \oddsidemargin -0.24cm % Définition taille de la page
   \evensidemargin -1.24cm \topskip 0cm \headheight -1.5cm % Définition des marges
   \RequirePackage{latexsym}                  % Symboles
   \RequirePackage{amsmath}                   % Symboles mathématiques
   \RequirePackage{tikz}   % Pour faire des schémas
   \RequirePackage{graphicx} % Pour inclure des images
   \RequirePackage{listings} % pour mettre des listings
% Fin Préambule; package qui peuvent être utiles

\title{Rapport de TP 4MMAOD : Génération de patch optimal}
\author{
DUHART Claudia (IF Groupe 1) 
\\ MARTINEZ Cléa (MMIS Groupe 1) 
}

\begin{document}

\maketitle
\parindent 0 mm 

%%%%%%%%%%%%%%%%%%%%%%%%%%%%%%%%%%%%%%%%%%%%%%
\section{Principe de notre  programme (1 point)}

Nous avons choisi d'implémenter la méthode itérative en stockant tous les coûts $V(i,j)$ dans un tableau. \\
Dans un premier temps, on parcourt les fichiers $F_1$ et $F_2$ pour stocker dans deux listes chaînées (puis des tableaux)  les données des fichiers ce qui permet entre autres de sauvegarder le nombre de caractères associé à chaque ligne. 
Ensuite, on calcule les coûts $V(i,j)$ qu'on stocke dans un tableau, ainsi que les opérations nécessaires pour arriver aux lignes i et j dans un tableau $Top$. \\
Un parcours de ces tableaux nous permet ensuite, via de nouveaux tableaux, de reconstruire le patch de coût minimal, et de l'afficher sur la sortie standard. On affiche également sur la sortie standard le cpût du patch minimal trouvé. \\
%%%%%%%%%%%%%%%%%%%%%%%%%%%%%%%%%%%%%%%%%%%%%%
\section{Analyse du coût théorique (3 points)}


  \subsection{Nombre  d'opérations en pire cas\,: }
  
  Le nombre d'opérations en pire cas est en $O(n_1 n_2 + c_1c_2)$
    \paragraph{Justification\,: \\ }
    

        La fonction \it initList \rm appelée  réalise $4$ affectations : $O(1)$ \\
        La fonction \it getNbCar \rm fait un accès tableau et une affectation : $O(1)$ \\
        La fonction \it minimum \rm fait appel à \it compareLignes \rm qui fait le nombre de caractères de la ligne fois un test d'égalité. 
        
        La fonction \it listLines \rm (appelée deux fois au total) fait $n$ appels à \it addNext \rm (fonction en $O(1)$ ) : $O(n)$ \\
        La vérification de l'allocation des tableaux \it Tab/Top \rm fait $n_1$ tests d'égalité (car les tableaux sont de taille $n_1$). \\
        Le remplissage des tableaux \it Tab et Top fait $n_1\times n_2$ (double boucle) appels à la fonction \it minimum : $O(c_1 \times c_2)$.\\
        La fonction \it build path \rm fait dans le pire des cas $O(max(n_1,n_2))$ affectations. \\
        La fonction \it printpath \rm fait au maximum $O(n_1+n_2)$ tests d'égalité  (car la taille du tableau path vaut $= n_1+n_2+2$ ) \\
   
  \subsection{Place mémoire requise\,: }
  
  La place mémoire requise est en $O(n_1 \times n_2)$. 
    \paragraph{Justification\,:  }
    
    On utilise deux tableaux de taille $(n_1+1)\times (n_2+1)$ (\it Tab \rm et \it Top \rm pour stocker les $V(i,j)$ et les opérations). \\
    On utilise également un tableau de taille $n_1+n_2+2$ pour stocker les opérations conduisant à la construction d'un patch de coût minimal, puis un tableau contenant les minimums des colonnes de \it Tab \rm de taille $ n_2+1 $\\
    Enfin, on stocke dans deux listes chainées( puis des tableaux)de tailles respectives $n_1 +1$ et $n_2+1$ les lignes des fichiers d'entrée et de sortie (de structure \it struct line \rm ).

  \subsection{Nombre de défauts de cache sur le modèle CO\,: }
   
Notons Z la taille du cache, L la taille des lignes de cache.\\  
Si $Z<max(n_1,n_2)/L$, l'initialisation des tableaux \it Tab \rm et \it Top \rm provoque environ $2(n_1+n_2)/L$ défauts de cache, sinon 0 défauts de cache. \\

Si $Z>2(n_1n_2)/L$, alors le remplissage des tableaux ne provoque pas de défauts de cache. Sinon, il en provoque $O(n_1n_2/L)$.

Si $Z > n_1+n_2+2$   le remplissage du tableau \it path \rm ne provoque pa de défaut de cache, sinon il y en a $O((n1+n2)/L)$.
%%%%%%%%%%%%%%%%%%%%%%%%%%%%%%%%%%%%%%%%%%%%%%
\section{Compte rendu d'expérimentation (2 points)}
  \subsection{Conditions expérimentaless}
     
    \subsubsection{Description synthétique de la machine\,:} 
      {Processeur: inter CORE I5vPro\\
      Fréquence: 800.292 MHz\\
      Mémoire: 8Go\\
      Système d'exploitation: Ubuntu\\
       Seul firefox était ouvert au moment des mesures pendant les tests.
      } 

    \subsubsection{Méthode utilisée pour les mesures de temps\,: } 
      {
       Les mesures de temps ont été effectuées avec la lignes suivante:\\
       \verb+time ./bin/computePatchOpt benchmark/benchmark_n/source benchmark/benchmark_n/target > patch+ \\
       Le temps retourné était donc en seconde, nous avons retenu la partie user.
       Un même test a été exécuté 5 fois de suite pour un même benchmark.
      }

  \subsection{Mesures expérimentales}


    \begin{figure}[h]
      \begin{center}
        \begin{tabular}{|l||r||r|r|r||}
          \hline
          \hline
            & coût         & temps     & temps   & temps \\
            & du patch     & min       & max     & moyen \\
          \hline
          \hline
            benchmark1 &  2540    &  0.045   & 0.096    & 0.072    \\
          \hline
            benchmark2 &    3120  &   0.195  & 0.255    & 0.284    \\
          \hline
            benchmark3 &     809 &  0.522   &  0.653   &  0.584   \\
          \hline
            benchmark4 &    1708  &   1.272  & 1.331    &  1.309   \\
          \hline
            benchmark5 &    7553  &   2.262  & 2.793    &  2.387   \\
          \hline
            benchmark6 &    37027  &  12.339   &  12.718   & 12.496    \\
          \hline
          \hline
        \end{tabular}
        \caption{Mesures des temps minimum, maximum et moyen de 5 exécutions pour les 6 benchmarks en seconde.}
        \label{table-temps}
      \end{center}
    \end{figure}

\subsection{Analyse des résultats expérimentaux}
A chaque benchmark, le nombre de lignes augmente, par conséquent le nombre d'opération (dépendant du nombre de lignes et du nombre de caractères des fichiers source et cible) augmente également. De plus, les structures de données utilisées pour stocker les données sont forcément de plus grande taille, et donc le nombre de défauts de cache augmente de manière exponentielle !! 
Les résultats expérimentaux présentés dans le tableau ci-dessous montrent bien cette augmentation exponentielle. 


              \begin{figure}[h]
      \begin{center}
        \begin{tabular}{|l||r||r|r|r||}
          \hline
          \hline
            Benchmark & LLi misses & LLd misses \\
          \hline
          \hline
            benchmark1 &  1 123 & 140 872    \\
          \hline
            benchmark2 &  1 130   & 6 832 521 \\
          \hline
            benchmark3 &  1 122  & 16 728 779\\
          \hline
            benchmark4 &  1 128    & 31 335 061\\
          \hline
            benchmark5 &   1 131   & 49 768 505\\
          \hline
            benchmark6 &    1 134 & 217 237 598\\
          \hline
          \hline
        \end{tabular}
        \caption{Nombres de défauts de cache expérimentaux pour chaque benchmark}
        \label{table-temps}
      \end{center}
    \end{figure}
  
   
    Mesures effectuées avec cachegrind, selon la commande suivante:\\
    \verb+valgrind --tool=cachegrind computePatchOpt F1 F2+ 
 

%%%%%%%%%%%%%%%%%%%%%%%%%%%%%%%%%%%%%%%%%%%%%%
\section{Question\,: et  si le coût d'un patch était sa taille en octets ? (1 point)}
Si le coût d'un patch était sa taille en octet, on utiliserait la même méthode que dans notre programme, on changerait seulement l'équation de Bellman pour modéliser le problème (et donc la fonction \it mimimum \rm ) de notre programme. \\

La nouvelle équation de Bellman serait la suivante : 
\begin{align*}
V(n,m)= min ~~(~~  &  V(n-1,m-1), \\
& 4 + L(F_2(m))+V(n,m-1),\\
&  4+L(F_2(m))+V(n-1,m-1)\\
& 4 + V(n-1, m),\\
& min_{i\in [2,n]} (6+ V(n-i,m)) ~~~)
\end{align*}

avec $L(F_2(j))$ le nombre de caractères de la ligne j du fichier $F_2$, en comptant le caractère de fin de ligne  \\

Si l'on recopie une ligne, alors ça ne coûte rien et on est ramené au même problème, sur des fichiers de longueur $n-1$ et $m-1$. \\
Si l'on ajoute une ligne, on écrit dans le patch $4+L(F_2(j))$ caractères et on est ramené au même problème sur des fichiers de longueur $n$ et $m-1$. \\
Si l'on substitue une ligne, on écrit dans le patch $4+L(F_2(j))$ caractères et on est ramené au même problème sur des fichiers de longueur $n-1$ et $m-1$. \\
Si l'on supprime une ligne, on écrit dans le patch $4$ caractères et on est ramené au même problème sur des fichiers de longueur $n-1$ et  $m$.\\
Enfin si l'on supprime $i$ lignes ($i\leq n$), on écrit dans le patch $6$ caractères et on est ramené au même problème sur des fichiers de longueur $n-i$ et $m$. 

\section{Optimisations possibles}
Tout d'abord, nous stockons les lignes dans 2 structures différentes: liste chaînée puis tableau. Cela nous permet un accès plus facile aux valeurs mais augmente la place mémoire nécessaire. \\
Le stockage de tous les coûts dans notre tableau Tab n'est pas indispensable. En effet, lorsque l'on s'intéresse à une case nous n'avons besoin que de la ligne associée ainsi que la précédente pour les calculs. Ainsi, nous aurions donc pu diminuer la place occupée par ce tableau. 



\end{document}
%% Fin mise au format

